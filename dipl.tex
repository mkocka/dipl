\documentclass[oneside,a4paper,11pt]{report}
\usepackage[T1]{slovak}
\usepackage[utf8]{inputenc}
\usepackage{latexsym}
\usepackage{graphicx}
\usepackage{amsmath}
\usepackage{caption}
\usepackage{fancyhdr}
%\usepackage{picins}
\usepackage{times}
\usepackage{mathptmx}
\usepackage{url}
\usepackage{booktabs}
\usepackage{appendix}

\pagestyle{fancy}
\fancyhead{}
\fancyhead[C]{\leftmark}
\renewcommand{\chaptermark}[1]{\markboth{\thechapter.\ #1}{}}


\usepackage[Sonny]{fncychap}
\makeatletter
 \ChNameVar{\small}
 \ChNumVar{\LARGE}
 \ChTitleVar{\LARGE\centering}
 \ChRuleWidth{0.05pt}
 \ChNameUpperCase

\title{Štúdium premenných hviezd vo vysokoenergetickej části spektra}
\author{Matúš Kocka}



\begin{document}
%\begin{figure}
%\begin{center}
%\includegraphics[width=6cm]{the-white-dwarf}
%\end{center}
%\end{figure}
%\pagebreak
\tableofcontents

\addcontentsline{toc}{chapter}{\protect\numberline{}Úvod}
\chapter*{Úvod}

.... pindy o CVs a SNe 

\chapter{Kataklyzmické premenné hviezdy }
%\section{Úvod do kataklyzmických premenných}   ---- toto pridat ak sa to bude hodit
Kataklyzmické premenné hviezdy (CV) sú dvojhviezdne systémi s rotačnou periódou typicky menšou 
ako jeden deň. Jedná sa teda o blízke dvojhviezdy, kde je primárnou zložkou biely trpaslík 
akreujúci hmotu cez Rocheov lalok zo sekundárnej zložky dvohviezdneho systému, ktorou je 
typicky hviezda hlavnej postupnosti \cite{(Warner 1995)}. CV sa stali v poslednej dobe 
cieľom veľkého záujmu astrofyziky vysokých energií potom, ako ich kozmické misie ako RXTE, 
INTEGRAL, či SWIFT detekovali niekoľko desiatok v oblasti nad 20keV. 

Informácia, že CV sú pozorovateľné aj vo vysokých energiách elmag. spektra nemusí byť až 
tak prekvapivá, keď si uvedomíme, že akreovaný materiál sa v akréčnom disku 
okolo bieleho trpaslíka zahrieva na vysoké teploty a následne padá na jeho povrch. 
Avšak detekcia nad 20keV sa 
stala reálnou až s rozvojom technologie a príchodom dostatočne citlivých kozmických misií. 
Priekopníkom bol projekt NASA menom RXTE\footnote{RXTE bola vypustená 30. Decembra 1995 z mysu Ceneveral} 
(Rossi X-ray Timing Explorer).

V závislosti na intenzite magnetického poľa bieleho trpaslíka existujú tri možnosti dopadu hmoty na 
jeho povrch. V prípade, že biely trpaslík nemá, resp. má len malé magnetické pole, hmota zo
sekundárnej zložky tečie cez Rocheov lalok a vytvára akréčny disk okolo rovníka bieleho trpaslíka a 
následne dopadá na jeho rovníkovú oblasť. 

V prípade, že je magnetické pole väčšie ako $10^{7 - 9} G$ jedná sa o tzv. polari. Polari nesú meno na 
základe typickej, veľmi silnej polarizácie v optickej a infračervenej 
oblasti spektra. Magnetické pole polaru je natoľko silné, že dôjde k sinchronizácií obežnej a rotačnej 
periódy bieleho trpaslíka $(P_{orb} = P_{spin})$. Akréčny disk sa v dôsledku extrémneho magnetického poľa
nevytvorí, materiál je unášaný po magnetických siločiarach a následne dopadá na povrch bieleho trpaslíka.

Existujú však aj kataklyzmické premenné hviezdy, kde je magnetické pole bieleho trpaslíka dosť silné nato, 
aby v určitej vzdialenosti od jeho povrchu zničilo vnútornú časť akréčneho disku, avšak nie celý disk. 
S magnetickým poľom o intenzite tipicky $>10^{5 - 7} G$, teda niekde na ceste medzi polarmi a nemagnetickými 
CV sa nachádzajú tzv. intermediálne polari.  

Intermediálne polari tvoria len malý zlomok všetkých známich CV, avšak silne dominujú v zložke detekovaných
CV v röntgenovej oblasti spektra $(>20-50keV)$. 
   
 
\section{Magnetické kataklyzmické premenné hviezdy}
\subsection{Polary}
\subsection{Intermedialne polary}
\subsection{Galaktická populácia kataklyzmických premenných hviezd}
\label{GXRE}
\section{Stručná história röntgenoého pozorovania intermedialnych polarov}

\chapter{Teoretický model intermediálnych polarov}
\section{Brzdné žiarenie (thermal bremstalung)}




\section{PSR}
\section{Hmotnosť bieleho trpaslík}
\subsection{Pomocou kontinua}
\subsection{Pomocou K železných čiar}
\chapter{Spracovanie dát}
\section{INTEGRAL}
\section{XMM-Newton}
\chapter{Určenie hmotností vybraných IP}



%\chapter{Intermedialne polary}   




\begin{thebibliography}{99}
\addcontentsline{toc}{chapter}{\hspace*{6mm}Literatúra}
\bibitem {(Warner 1995)} Warner, B. 1995, Cataclysmic Variables, (Cambridge: Cambridge Univ. Press)


\end{thebibliography}


\appendix
\section*{Appendix}
teste teste
%\pagebreak
%\thispagestyle{empty}
%\LaTeX{}
\end{document}
